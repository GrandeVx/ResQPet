% ============================================
% SEZIONE 6: STRAY POSE CLASSIFIER (CONTRIBUTO ORIGINALE)
% ============================================

\section{Stray Pose Classifier}
\label{sec:pose}

\subsection{Contributo Originale: Weak Supervision}

Il Pose Classifier rappresenta il \textbf{contributo originale} di questo lavoro. Invece di richiedere annotazioni manuali costose e soggettive per classificare le posture come ``stray-like'' o ``owned-like'', proponiamo un approccio di \textbf{weak supervision} che sfrutta l'origine dei dati come supervisione implicita.

\begin{figure}[H]
    \centering
    % Weak Supervision Concept for Pose Classifier
\resizebox{\textwidth}{!}{%
\begin{tikzpicture}[
    node distance=1cm,
    box/.style={rectangle, draw, rounded corners, minimum width=2.2cm, minimum height=0.8cm, align=center, font=\small},
    arrow/.style={->, thick, >=stealth}
]

% Datasets
\node[box, fill=green!30] (owned) {Stanford Dogs\\(Padronali)};
\node[box, fill=red!30, below=1.2cm of owned] (stray) {FYP Stray Dogs\\(Randagi)};

% Labels
\node[box, fill=green!10, right=1.5cm of owned] (label0) {Label: 0\\(Non randagio)};
\node[box, fill=red!10, right=1.5cm of stray] (label1) {Label: 1\\(Randagio)};

% Keypoints
\node[box, fill=blue!20, right=1.5cm of label0, yshift=-0.6cm] (kpts) {24 Keypoints\\Normalizzati};

% Classifier
\node[box, fill=orange!30, right=1.2cm of kpts] (mlp) {MLP\\Classifier};

% Output
\node[box, fill=purple!20, right=1.2cm of mlp] (output) {P(stray)};

% Arrows
\draw[arrow] (owned) -- (label0);
\draw[arrow] (stray) -- (label1);
\draw[arrow] (label0) -- (kpts);
\draw[arrow] (label1) -- (kpts);
\draw[arrow] (kpts) -- (mlp);
\draw[arrow] (mlp) -- (output);

% Annotation
\node[below=0.3cm of kpts, font=\scriptsize, align=center] {Weak Label:\\origine dataset};

\end{tikzpicture}%
}

    \caption{Schema dell'approccio weak supervision. I label sono derivati automaticamente dall'origine del dataset: keypoints da FYP Dataset (cani randagi) ricevono label=1, keypoints da Stanford Dogs (cani padronali) ricevono label=0.}
    \label{fig:weak_supervision}
\end{figure}

\subsection{Motivazione}

L'annotazione manuale della postura presenta problemi significativi:

\begin{itemize}
    \item \textbf{Soggettivit\`a}: Cosa definisce una ``postura da randagio''?
    \item \textbf{Costo}: Annotare migliaia di pose richiede tempo e risorse
    \item \textbf{Inconsistenza}: Annotatori diversi producono label diversi
    \item \textbf{Scalabilit\`a}: Difficile estendere a nuovi dataset
\end{itemize}

Il nostro approccio risolve questi problemi derivando i label automaticamente dalla provenienza delle immagini.

\subsection{Approccio Weak Supervision}

\subsubsection{Assunzione Fondamentale}

\begin{quote}
\textit{I cani fotografati in dataset di cani randagi (es. FYP Dataset) tendono ad avere posture diverse dai cani fotografati in contesti domestici/esposizioni (es. Stanford Dogs).}
\end{quote}

Questa assunzione si basa sul fatto che:
\begin{itemize}
    \item I cani randagi mostrano spesso comportamenti difensivi, sottomessi o stressati
    \item I cani padronali in foto sono tipicamente rilassati, in posa, o in attivit\`a ludiche
\end{itemize}

\subsubsection{Schema di Labeling}

\begin{table}[H]
\centering
\begin{tabular}{lll}
\toprule
\textbf{Dataset Origine} & \textbf{Label} & \textbf{Motivazione} \\
\midrule
FYP Dataset & 1 (Stray) & Cani randagi di strada \\
Stanford Dogs & 0 (Owned) & Cani in contesti domestici/show \\
Dog's Skin Diseases & 0 (Owned) & Cani con proprietari (dal veterinario) \\
\bottomrule
\end{tabular}
\caption{Schema di assegnazione automatica dei label basato sull'origine del dataset.}
\label{tab:weak_labels}
\end{table}

\subsubsection{Pipeline di Estrazione}

\begin{enumerate}
    \item Per ogni immagine nei dataset:
    \begin{enumerate}
        \item Eseguire YOLO11 Dog-Pose per detection
        \item Filtrare detection con confidence $> 0.7$
        \item Estrarre keypoints e normalizzare rispetto a bbox
        \item Assegnare label basato su origine dataset
    \end{enumerate}
    \item Bilanciare le classi (stesso numero di campioni)
    \item Split train/val/test stratificato
\end{enumerate}

\subsection{Architettura MLP}

Il classificatore utilizza un Multi-Layer Perceptron leggero, ottimale per input numerici strutturati.

\begin{figure}[H]
    \centering
    % MLP Architecture for Pose Classifier
\begin{tikzpicture}[
    node distance=1.2cm,
    neuron/.style={circle, draw, minimum size=0.4cm, fill=blue!20},
    layer/.style={rectangle, draw, rounded corners, minimum width=1cm, minimum height=2.5cm, fill=gray!10},
    arrow/.style={->, thick, >=stealth}
]

% Input layer
\node[layer, minimum height=3cm] (input) {};
\node[above=0.1cm of input, font=\small\bfseries] {Input};
\node[below=0.1cm of input, font=\scriptsize] {72};
\foreach \i in {1,...,5} {
    \node[neuron] at ($(input.center) + (0, 1.2-0.5*\i)$) {};
}
\node at ($(input.center) + (0, -0.8)$) {$\vdots$};

% Hidden 1
\node[layer, right=1.5cm of input, minimum height=2.5cm] (h1) {};
\node[above=0.1cm of h1, font=\small\bfseries] {Hidden 1};
\node[below=0.1cm of h1, font=\scriptsize] {128};
\foreach \i in {1,...,4} {
    \node[neuron, fill=green!30] at ($(h1.center) + (0, 0.9-0.5*\i)$) {};
}
\node at ($(h1.center) + (0, -0.6)$) {$\vdots$};

% Hidden 2
\node[layer, right=1.5cm of h1, minimum height=2cm] (h2) {};
\node[above=0.1cm of h2, font=\small\bfseries] {Hidden 2};
\node[below=0.1cm of h2, font=\scriptsize] {64};
\foreach \i in {1,...,3} {
    \node[neuron, fill=orange!30] at ($(h2.center) + (0, 0.6-0.5*\i)$) {};
}
\node at ($(h2.center) + (0, -0.5)$) {$\vdots$};

% Output
\node[layer, right=1.5cm of h2, minimum height=1cm, minimum width=0.8cm] (output) {};
\node[above=0.1cm of output, font=\small\bfseries] {Output};
\node[below=0.1cm of output, font=\scriptsize] {1};
\node[neuron, fill=red!30] at (output.center) {};

% Connections (simplified)
\draw[arrow, gray] (input.east) -- (h1.west);
\draw[arrow, gray] (h1.east) -- (h2.west);
\draw[arrow, gray] (h2.east) -- (output.west);

% Activations
\node[below=0.8cm of h1, font=\scriptsize] {ReLU};
\node[below=0.8cm of h2, font=\scriptsize] {ReLU};
\node[below=0.5cm of output, font=\scriptsize] {Sigmoid};

\end{tikzpicture}

    \caption{Architettura del Pose Classifier MLP. Input: 72 features (24 keypoints $\times$ 3 valori). Output: probabilit\`a di postura stray-like.}
    \label{fig:mlp_pose}
\end{figure}

\subsubsection{Dettaglio Architettura}

\begin{lstlisting}[style=python, caption={Definizione architettura StrayPoseMLP}]
class StrayPoseMLP(nn.Module):
    def __init__(self, input_dim=72, hidden_dims=[128, 64], dropout=0.3):
        super().__init__()

        layers = []
        prev_dim = input_dim  # 72 = 24 keypoints * 3

        for hidden_dim in hidden_dims:
            layers.extend([
                nn.Linear(prev_dim, hidden_dim),
                nn.ReLU(),
                nn.BatchNorm1d(hidden_dim),
                nn.Dropout(dropout)
            ])
            prev_dim = hidden_dim

        # Output layer
        layers.append(nn.Linear(prev_dim, 1))
        layers.append(nn.Sigmoid())

        self.model = nn.Sequential(*layers)

    def forward(self, x):
        return self.model(x)  # P(stray_pose)
\end{lstlisting}

\subsubsection{Dimensioni Tensori}

\begin{table}[H]
\centering
\begin{tabular}{lcl}
\toprule
\textbf{Layer} & \textbf{Output Shape} & \textbf{Parametri} \\
\midrule
Input & (B, 72) & -- \\
Linear + ReLU + BN + Dropout & (B, 128) & $72 \times 128 + 128 = 9,344$ \\
Linear + ReLU + BN + Dropout & (B, 64) & $128 \times 64 + 64 = 8,256$ \\
Linear + Sigmoid & (B, 1) & $64 \times 1 + 1 = 65$ \\
\midrule
\textbf{Totale} & -- & \textbf{$\sim$17,665} \\
\bottomrule
\end{tabular}
\caption{Dimensioni dei tensori e numero di parametri per layer.}
\label{tab:mlp_dims}
\end{table}

\subsection{Feature Input: Keypoints Normalizzati}

L'input al classificatore \`e un vettore di 72 features:

\begin{equation}
    \mathbf{x} = [x_0, y_0, v_0, x_1, y_1, v_1, \ldots, x_{23}, y_{23}, v_{23}]
\end{equation}

dove per ogni keypoint $i$:
\begin{itemize}
    \item $x_i \in [0, 1]$: coordinata x normalizzata rispetto a bbox
    \item $y_i \in [0, 1]$: coordinata y normalizzata rispetto a bbox
    \item $v_i \in [0, 1]$: visibility score
\end{itemize}

\subsection{Indicatori Comportamentali}

Il modello impara implicitamente a riconoscere pattern posturali associati a stress/abbandono:

\begin{table}[H]
\centering
\begin{tabular}{ll}
\toprule
\textbf{Indicatore} & \textbf{Keypoints Coinvolti} \\
\midrule
Coda tra le gambe & tail\_start, tail\_end, back\_paws \\
Testa bassa & nose, chin, throat vs withers \\
Orecchie appiattite & ear\_base, ear\_tip \\
Postura accucciata & rapporto altezza/larghezza corpo \\
Stance difensiva & distanza tra front\_paws e back\_paws \\
Asimmetria posturale & confronto lato sinistro/destro \\
\bottomrule
\end{tabular}
\caption{Indicatori comportamentali impliciti catturati dal modello attraverso i keypoints.}
\label{tab:pose_indicators}
\end{table}

\subsection{Dataset}

\begin{table}[H]
\centering
\begin{tabular}{ll}
\toprule
\textbf{Propriet\`a} & \textbf{Valore} \\
\midrule
Origine Stray (label=1) & FYP Dataset \\
Origine Owned (label=0) & Stanford Dogs + Skin Dataset \\
Campioni totali & 15,132 (bilanciati) \\
Training & 70\% \\
Validation & 15\% \\
Test & 15\% \\
\bottomrule
\end{tabular}
\caption{Composizione del dataset per il Pose Classifier.}
\label{tab:pose_dataset}
\end{table}

\subsection{Training}

\begin{table}[H]
\centering
\begin{tabular}{ll}
\toprule
\textbf{Parametro} & \textbf{Valore} \\
\midrule
Epochs & 100 \\
Batch size & 64 \\
Optimizer & Adam \\
Learning rate & $10^{-3}$ \\
Weight decay & $10^{-4}$ \\
Loss & BCELoss \\
Scheduler & ReduceLROnPlateau \\
Early stopping & 15 epochs \\
\bottomrule
\end{tabular}
\caption{Configurazione training del Pose Classifier.}
\label{tab:pose_training}
\end{table}

\subsection{Vantaggi dell'Approccio}

\begin{enumerate}
    \item \textbf{Nessuna annotazione manuale}: I label derivano automaticamente
    \item \textbf{Scalabilit\`a}: Facilmente estendibile con nuovi dataset
    \item \textbf{Consistenza}: Eliminata la soggettivit\`a umana
    \item \textbf{Dataset ampio}: Potenzialmente decine di migliaia di campioni
    \item \textbf{Riproducibilit\`a}: Processo completamente automatico
\end{enumerate}

\subsection{Limitazioni}

\begin{itemize}
    \item \textbf{Label noise}: Non tutti i cani del FYP mostrano posture ``stray''
    \item \textbf{Overlap}: Alcune posture sono comuni a entrambe le categorie
    \item \textbf{Bias del dataset}: La qualit\`a dipende dalla rappresentativit\`a dei dataset
\end{itemize}

\subsection{Metriche}

\begin{table}[H]
\centering
\begin{tabular}{lcc}
\toprule
\textbf{Metrica} & \textbf{Target} & \textbf{Ottenuto} \\
\midrule
Accuracy & $> 0.70$ & 0.589 \\
AUC-ROC & $> 0.75$ & 0.633 \\
\bottomrule
\end{tabular}
\caption{Metriche di valutazione del Pose Classifier. I risultati sono inferiori ai target, come atteso per approcci weak-supervised.}
\label{tab:pose_metrics}
\end{table}

\subsection{Analisi dei Risultati e Discussione}

I risultati ottenuti (Accuracy 0.589, AUC 0.633) sono inferiori ai target prefissati. Questo \`e un risultato \textbf{atteso} e documentato nella letteratura sulla weak supervision:

\begin{itemize}
    \item \textbf{Rumore intrinseco nei label}: L'assunzione che l'origine del dataset determini la postura \`e debole. Un cane randagio pu\`o essere fotografato in pose rilassate, cos\`i come un cane padronale pu\`o mostrare stress.

    \item \textbf{Overlap delle distribuzioni}: Le posture di cani randagi e padronali si sovrappongono significativamente, limitando la separabilit\`a delle classi.

    \item \textbf{Segnale comunque informativo}: Un AUC di 0.633 indica che il modello cattura un segnale, seppur debole, che pu\`o contribuire alla decisione finale quando combinato con gli altri classificatori.
\end{itemize}

\begin{figure}[H]
    \centering
    \includegraphics[width=\textwidth]{figures/keypoints_analysis.png}
    \caption{Analisi delle distribuzioni dei keypoints per classe (Owned vs Stray). In alto a sinistra: distribuzione della confidence di detection. In alto a destra: scatter plot della posizione del naso normalizzata. In basso a sinistra: visibilit\`a media dei keypoints. In basso a destra: boxplot della posizione Y del naso. Si nota il \textbf{significativo overlap} tra le due classi, che spiega le performance limitate della weak supervision.}
    \label{fig:keypoints_analysis}
\end{figure}

\subsubsection{Decisione: Mantenimento del Componente}

Nonostante i risultati sotto-target, si \`e deciso di \textbf{mantenere} il Pose Classifier nel sistema di fusione per le seguenti ragioni:

\begin{enumerate}
    \item \textbf{Contributo marginale positivo}: Con peso del 25\% nella fusione, anche un segnale imperfetto pu\`o spostare lo Stray Index nella direzione corretta
    \item \textbf{Complementarit\`a}: La postura cattura informazione diversa da collare, pelle e razza
    \item \textbf{Costo computazionale nullo}: Il calcolo avviene sui keypoints gi\`a estratti dal backbone, senza overhead
    \item \textbf{Comportamento conservativo}: In caso di incertezza (predizione $\sim 0.5$), il contributo si annulla naturalmente
\end{enumerate}

\paragraph{Alternativa considerata:} Ridurre il peso dal 25\% al 15\%, redistribuendo il 10\% al Collar Detector (35\% $\rightarrow$ 40\%) e Skin Classifier (20\% $\rightarrow$ 25\%). Questa opzione rimane disponibile per future ottimizzazioni.
