% ============================================
% SEZIONE 2: ARCHITETTURA DEL SISTEMA
% ============================================

\section{Architettura del Sistema}
\label{sec:architettura}

\subsection{Overview della Pipeline}

Il sistema ResQPet implementa un'architettura a pipeline multi-stadio per l'identificazione dello stato di abbandono nei cani. La pipeline si compone di tre fasi principali:

\begin{enumerate}
    \item \textbf{Detection e Pose Estimation}: Un modello YOLO11 specializzato rileva i cani nel frame ed estrae 24 keypoints anatomici
    \item \textbf{Classificazione Parallela}: Quattro classificatori indipendenti analizzano aspetti diversi del cane rilevato
    \item \textbf{Fusione Pesata}: Le probabilità dei classificatori vengono combinate in uno \textit{Stray Index} finale
\end{enumerate}

\begin{figure}[H]
    \centering
    \resizebox{0.95\textwidth}{!}{%
        % Pipeline Overview - ResQPet System Architecture
\begin{tikzpicture}[
    node distance=1.2cm and 1.8cm,
    box/.style={rectangle, draw, rounded corners, minimum width=2cm, minimum height=0.9cm, align=center, fill=blue!10},
    arrow/.style={->, thick, >=stealth}
]

% Input
\node[box, fill=green!20] (input) {Frame\\Input};

% Backbone
\node[box, right=of input] (backbone) {Backbone\\YOLO11};

% Classifiers - disposti verticalmente con maggiore spaziatura
\node[box, right=2.2cm of backbone, yshift=2.4cm] (collar) {Collar\\Detector};
\node[box, right=2.2cm of backbone, yshift=0.8cm] (skin) {Skin\\Classifier};
\node[box, right=2.2cm of backbone, yshift=-0.8cm] (pose) {Pose\\Classifier};
\node[box, right=2.2cm of backbone, yshift=-2.4cm] (breed) {Breed\\Classifier};

% Fusion
\node[box, right=2cm of skin, yshift=-0.5cm, fill=orange!20] (fusion) {Fusion\\Module};

% Output
\node[box, right=of fusion, fill=red!20] (output) {Stray\\Index};

% Arrows from input to backbone
\draw[arrow] (input) -- (backbone);

% Arrows from backbone to classifiers
\draw[arrow] (backbone.east) -- ++(0.5,0) |- (collar.west);
\draw[arrow] (backbone.east) -- ++(0.5,0) |- (skin.west);
\draw[arrow] (backbone.east) -- ++(0.5,0) |- (pose.west);
\draw[arrow] (backbone.east) -- ++(0.5,0) |- (breed.west);

% Arrows from classifiers to fusion
\draw[arrow] (collar.east) -| (fusion.north);
\draw[arrow] (skin.east) -- (fusion.west |- skin.east) -- (fusion.west);
\draw[arrow] (pose.east) -- (fusion.west |- pose.east) -- (fusion.west);
\draw[arrow] (breed.east) -| (fusion.south);

% Arrow from fusion to output
\draw[arrow] (fusion) -- (output);

% Labels
\node[below=0.2cm of backbone, font=\scriptsize] {ROI + Keypoints};
\node[above=0.1cm of fusion, font=\scriptsize] {Weighted Sum};

\end{tikzpicture}

    }%
    \caption{Architettura complessiva del sistema ResQPet. Il backbone YOLO11 estrae bounding box e keypoints, che vengono poi elaborati da quattro classificatori specializzati. Le probabilità risultanti sono combinate attraverso fusione pesata per produrre lo Stray Index finale.}
    \label{fig:pipeline}
\end{figure}

\subsection{Flusso dei Dati}

Il flusso di elaborazione procede come segue:

\begin{enumerate}
    \item \textbf{Input}: Frame video/immagine dalla sorgente CCTV
    \item \textbf{Backbone}: YOLO11 Dog-Pose rileva ogni cane e produce:
    \begin{itemize}
        \item Bounding box $[x_1, y_1, x_2, y_2]$
        \item 24 keypoints anatomici con confidence
        \item ROI (Region of Interest) croppata
    \end{itemize}
    \item \textbf{Pre-processing}:
    \begin{itemize}
        \item Normalizzazione keypoints rispetto alla bbox
        \item Resize ROI per i classificatori (224$\times$224)
    \end{itemize}
    \item \textbf{Classificazione Parallela}: Per ogni cane rilevato:
    \begin{itemize}
        \item Collar Detector $\rightarrow P(\text{no\_collar})$
        \item Skin Classifier $\rightarrow P(\text{disease})$
        \item Pose Classifier $\rightarrow P(\text{stray\_pose})$
        \item Breed Classifier $\rightarrow P(\text{stray}|\text{breed})$
    \end{itemize}
    \item \textbf{Fusione}: Combinazione pesata delle quattro probabilit\`a
    \item \textbf{Output}: Stray Index $\in [0, 1]$ con classificazione
\end{enumerate}

\subsection{Moduli di Classificazione}

I quattro classificatori sono progettati per catturare aspetti complementari che indicano lo stato di abbandono:

\begin{table}[H]
\centering
\begin{tabular}{llcc}
\toprule
\textbf{Modulo} & \textbf{Architettura} & \textbf{Input} & \textbf{Peso} \\
\midrule
Collar Detector & YOLOv8n & ROI 640$\times$640 & 35\% \\
Skin Classifier & ResNet50 & ROI 224$\times$224 & 20\% \\
Pose Classifier & MLP & 72 features & 25\% \\
Breed Classifier & EfficientNet-B0 & ROI 224$\times$224 & 20\% \\
\bottomrule
\end{tabular}
\caption{Riepilogo dei moduli di classificazione con relative architetture, input e pesi nella fusione.}
\label{tab:moduli}
\end{table}

\subsubsection{Collar Detector}
Rileva la presenza di collare, pettorina o guinzaglio. L'assenza di accessori \`e un forte indicatore di cane randagio, motivo per cui questo modulo ha il peso maggiore (35\%).

\subsubsection{Skin Classifier}
Identifica patologie cutanee indicative di trascuratezza o abbandono prolungato. Le malattie della pelle non curate suggeriscono mancanza di cure veterinarie.

\subsubsection{Pose Classifier}
Analizza la postura del cane basandosi sui keypoints estratti. Utilizza un approccio di \textbf{weak supervision} (descritto in dettaglio nella Sezione \ref{sec:pose}) per identificare posture tipiche di cani randagi (coda tra le gambe, testa bassa, postura difensiva).

\subsubsection{Breed Classifier}
Identifica la razza del cane per applicare prior statistici. Alcune razze sono statisticamente pi\`u rappresentate nei canili italiani (es. Pitbull, meticci) rispetto ad altre (es. Retriever, razze toy).

\subsection{Sistema di Fusione}
\label{sec:fusion_overview}

Le probabilit\`a dei quattro classificatori vengono combinate attraverso una media pesata:

\begin{equation}
    \text{Stray Index} = w_c \cdot P_c + w_s \cdot P_s + w_p \cdot P_p + w_b \cdot P_b
\end{equation}

dove:
\begin{itemize}
    \item $P_c$ = probabilit\`a di assenza collare
    \item $P_s$ = probabilit\`a di malattia cutanea
    \item $P_p$ = probabilit\`a di postura stray-like
    \item $P_b$ = prior di abbandono dato la razza
    \item $w_c = 0.35$, $w_s = 0.20$, $w_p = 0.25$, $w_b = 0.20$
\end{itemize}

Lo Stray Index risultante viene classificato in tre categorie:

\begin{table}[H]
\centering
\begin{tabular}{cll}
\toprule
\textbf{Range} & \textbf{Classificazione} & \textbf{Colore} \\
\midrule
$[0.0, 0.3)$ & Padronale & \colorbox{green!30}{Verde} \\
$[0.3, 0.7)$ & Possibile Smarrito & \colorbox{yellow!50}{Giallo} \\
$[0.7, 1.0]$ & Probabile Randagio & \colorbox{red!30}{Rosso} \\
\bottomrule
\end{tabular}
\caption{Soglie di classificazione dello Stray Index.}
\label{tab:soglie}
\end{table}
