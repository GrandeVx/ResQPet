% ============================================
% SEZIONE 5: SKIN DISEASE CLASSIFIER
% ============================================

\section{Skin Disease Classifier}
\label{sec:skin}

\subsection{Ruolo nel Sistema}

Lo Skin Classifier identifica patologie cutanee che possono indicare trascuratezza o abbandono prolungato. Un cane con malattie della pelle non curate suggerisce mancanza di cure veterinarie, tipica di animali randagi.

\begin{itemize}
    \item \textbf{Input}: ROI del cane ridimensionata a 224$\times$224 pixel
    \item \textbf{Output}: $P(\text{disease}) \in [0, 1]$
    \item \textbf{Peso Fusion}: 20\%
\end{itemize}

\subsection{Architettura}

Il classificatore utilizza ResNet50 pre-trainato su ImageNet con un custom classification head.

\begin{figure}[H]
    \centering
    % ResNet50 Residual Block
\begin{tikzpicture}[
    node distance=0.8cm,
    box/.style={rectangle, draw, minimum width=2cm, minimum height=0.6cm, align=center, font=\small},
    arrow/.style={->, thick, >=stealth}
]

% Main path
\node[box, fill=blue!20] (input) {Input};
\node[box, fill=green!20, below=of input] (conv1) {Conv 1$\times$1};
\node[box, fill=green!20, below=of conv1] (bn1) {BatchNorm + ReLU};
\node[box, fill=green!20, below=of bn1] (conv2) {Conv 3$\times$3};
\node[box, fill=green!20, below=of conv2] (bn2) {BatchNorm + ReLU};
\node[box, fill=green!20, below=of bn2] (conv3) {Conv 1$\times$1};
\node[box, fill=green!20, below=of conv3] (bn3) {BatchNorm};

% Addition
\node[circle, draw, below=of bn3, minimum size=0.6cm] (add) {$+$};
\node[box, fill=orange!20, below=of add] (relu) {ReLU};
\node[box, fill=blue!20, below=of relu] (output) {Output};

% Skip connection
\draw[arrow] (input) -- (conv1);
\draw[arrow] (conv1) -- (bn1);
\draw[arrow] (bn1) -- (conv2);
\draw[arrow] (conv2) -- (bn2);
\draw[arrow] (bn2) -- (conv3);
\draw[arrow] (conv3) -- (bn3);
\draw[arrow] (bn3) -- (add);
\draw[arrow] (add) -- (relu);
\draw[arrow] (relu) -- (output);

% Skip connection arrow
\draw[arrow, dashed, red] (input.east) -- ++(1.5,0) |- (add.east);
\node[right=1.8cm of conv2, font=\scriptsize, text=red] {Skip};

\end{tikzpicture}

    \caption{Blocco residuo di ResNet. La skip connection permette di trainare reti profonde evitando il problema del vanishing gradient.}
    \label{fig:resnet_block}
\end{figure}

\subsubsection{Struttura Completa}

\begin{enumerate}
    \item \textbf{Backbone}: ResNet50 (senza ultimo layer FC)
    \begin{itemize}
        \item Pre-trained su ImageNet (1000 classi)
        \item Output: 2048 features
    \end{itemize}
    \item \textbf{Classification Head}:
    \begin{itemize}
        \item Dropout(0.3)
        \item Linear(2048 $\rightarrow$ 256)
        \item ReLU
        \item BatchNorm1d(256)
        \item Dropout(0.3)
        \item Linear(256 $\rightarrow$ 6)
    \end{itemize}
\end{enumerate}

\subsubsection{Diagramma Architettura}

\begin{lstlisting}[style=python, caption={Definizione architettura Skin Classifier}]
class SkinClassifier(nn.Module):
    def __init__(self, num_classes=6):
        super().__init__()
        # Backbone ResNet50
        self.backbone = timm.create_model(
            'resnet50',
            pretrained=True,
            num_classes=0  # Rimuove FC layer
        )
        # Custom head
        self.classifier = nn.Sequential(
            nn.Dropout(0.3),
            nn.Linear(2048, 256),
            nn.ReLU(),
            nn.BatchNorm1d(256),
            nn.Dropout(0.3),
            nn.Linear(256, num_classes)
        )

    def forward(self, x):
        features = self.backbone(x)  # (B, 2048)
        return self.classifier(features)  # (B, 6)
\end{lstlisting}

\subsection{Dataset}

Il modello \`e trainato sul dataset ``Dog's Skin Diseases'' disponibile su Kaggle.

\begin{table}[H]
\centering
\begin{tabular}{ll}
\toprule
\textbf{Propriet\`a} & \textbf{Valore} \\
\midrule
Nome & Dog's Skin Diseases \\
Fonte & Kaggle \\
Totale immagini & 4,315 \\
Training set & 3,022 (70\%) \\
Validation set & 860 (20\%) \\
Test set & 433 (10\%) \\
Classi & 6 \\
\bottomrule
\end{tabular}
\caption{Statistiche del dataset Skin Diseases.}
\label{tab:skin_dataset}
\end{table}

\subsubsection{Classi Patologiche}

\begin{table}[H]
\centering
\begin{tabular}{lcc}
\toprule
\textbf{Classe} & \textbf{Descrizione} & \textbf{Peso Malattia} \\
\midrule
Healthy & Pelle sana & 0.0 \\
Dermatitis & Infiammazione generica & 0.6 \\
Fungal\_infections & Infezioni fungine & 0.7 \\
Hypersensitivity & Allergie/ipersensibilit\`a & 0.4 \\
Demodicosis & Acari Demodex & 0.8 \\
Ringworm & Tigna (tinea) & 0.75 \\
\bottomrule
\end{tabular}
\caption{Classi del dataset Skin Diseases con relativi pesi di gravit\`a. Il peso indica quanto la patologia contribuisce al P(disease).}
\label{tab:skin_classes}
\end{table}

\subsubsection{Calcolo P(disease)}

La probabilit\`a finale di malattia \`e calcolata come somma pesata delle probabilit\`a per classe:

\begin{lstlisting}[style=python, caption={Calcolo della probabilit\`a di malattia cutanea}]
DISEASE_WEIGHTS = {
    'Healthy': 0.0,
    'Dermatitis': 0.6,
    'Fungal_infections': 0.7,
    'Hypersensitivity': 0.4,
    'Demodicosis': 0.8,
    'Ringworm': 0.75
}

def compute_p_disease(class_probabilities):
    p_disease = 0.0
    for class_name, prob in class_probabilities.items():
        p_disease += prob * DISEASE_WEIGHTS[class_name]
    return p_disease
\end{lstlisting}

\subsection{Data Augmentation}

\begin{table}[H]
\centering
\begin{tabular}{ll}
\toprule
\textbf{Trasformazione} & \textbf{Parametri} \\
\midrule
Resize & 256 $\times$ 256 \\
RandomCrop & 224 $\times$ 224 \\
Horizontal Flip & 50\% \\
Vertical Flip & 20\% \\
Rotazione & $\pm 15$\textdegree \\
ColorJitter & brightness=0.2, contrast=0.2 \\
Normalize & ImageNet mean/std \\
\bottomrule
\end{tabular}
\caption{Data augmentation per il training dello Skin Classifier.}
\label{tab:skin_aug}
\end{table}

\subsection{Training}

Il training utilizza un approccio di fine-tuning progressivo:

\begin{enumerate}
    \item \textbf{Fase 1}: Freeze backbone, train solo classification head (10 epochs)
    \item \textbf{Fase 2}: Unfreeze ultimi 2 blocchi ResNet (20 epochs)
    \item \textbf{Fase 3}: Fine-tuning completo con LR ridotto (20 epochs)
\end{enumerate}

\begin{table}[H]
\centering
\begin{tabular}{ll}
\toprule
\textbf{Parametro} & \textbf{Valore} \\
\midrule
Epochs totali & 50 \\
Batch size & 32 \\
Optimizer & AdamW \\
Learning rate & $10^{-4}$ \\
Weight decay & $10^{-5}$ \\
Loss & CrossEntropyLoss (class weights) \\
Scheduler & ReduceLROnPlateau \\
Early stopping & 10 epochs \\
\bottomrule
\end{tabular}
\caption{Configurazione training dello Skin Classifier.}
\label{tab:skin_training}
\end{table}

\subsubsection{Class Weights}

Per gestire lo sbilanciamento del dataset, utilizziamo pesi inversamente proporzionali alla frequenza:

\begin{equation}
    w_i = \frac{\max_j(n_j)}{n_i}
\end{equation}

dove $n_i$ \`e il numero di campioni della classe $i$.

\subsection{Metriche}

\begin{table}[H]
\centering
\begin{tabular}{lcc}
\toprule
\textbf{Metrica} & \textbf{Target} & \textbf{Ottenuto} \\
\midrule
Accuracy & $> 0.80$ & -- \\
F1-Score (macro) & $> 0.75$ & -- \\
Precision (macro) & $> 0.75$ & -- \\
Recall (macro) & $> 0.70$ & -- \\
\bottomrule
\end{tabular}
\caption{Metriche di valutazione dello Skin Classifier.}
\label{tab:skin_metrics}
\end{table}
