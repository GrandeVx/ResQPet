% ============================================
% SEZIONE 3: YOLO11 DOG-POSE BACKBONE
% ============================================

\section{YOLO11 Dog-Pose Backbone}
\label{sec:backbone}

\subsection{Ruolo nel Sistema}

Il backbone rappresenta il primo stadio della pipeline e ha il compito critico di:
\begin{itemize}
    \item \textbf{Rilevare} tutti i cani presenti nel frame
    \item \textbf{Localizzare} ciascun cane con una bounding box precisa
    \item \textbf{Estrarre} 24 keypoints anatomici per l'analisi della postura
\end{itemize}

A differenza del modello YOLO11-pose standard (trainato per pose estimation umana con 17 keypoints), il nostro backbone \`e specializzato per la detection di cani con 24 punti anatomici specifici.

\subsection{Architettura YOLO11}

YOLO11 (You Only Look Once, versione 11) \`e l'ultima evoluzione della famiglia YOLO, sviluppata da Ultralytics. L'architettura si compone di tre blocchi principali:

\begin{figure}[H]
    \centering
    % YOLO11 Architecture for Dog Pose Estimation
\begin{tikzpicture}[
    node distance=1.2cm,
    box/.style={rectangle, draw, rounded corners, minimum width=2cm, minimum height=0.8cm, align=center, font=\small},
    arrow/.style={->, thick, >=stealth}
]

% Backbone
\node[box, fill=blue!20] (input) {Input\\640$\times$640};
\node[box, fill=blue!30, right=of input] (csp) {CSPDarknet\\Backbone};

% Neck
\node[box, fill=green!20, right=of csp] (panet) {PANet\\Neck};

% Heads
\node[box, fill=orange!20, above right=0.5cm and 1cm of panet] (det) {Detection\\Head};
\node[box, fill=red!20, below right=0.5cm and 1cm of panet] (pose) {Pose\\Head};

% Outputs
\node[box, fill=gray!20, right=of det] (bbox) {BBox\\+ Conf};
\node[box, fill=gray!20, right=of pose] (kpts) {24 Keypoints\\(x,y,v)};

% Arrows
\draw[arrow] (input) -- (csp);
\draw[arrow] (csp) -- (panet);
\draw[arrow] (panet) -- (det);
\draw[arrow] (panet) -- (pose);
\draw[arrow] (det) -- (bbox);
\draw[arrow] (pose) -- (kpts);

% Feature pyramids
\node[below=0.2cm of csp, font=\scriptsize] {Multi-scale features};
\node[below=0.2cm of panet, font=\scriptsize] {Feature fusion};

\end{tikzpicture}

    \caption{Architettura semplificata di YOLO11 per pose estimation. Il backbone CSPDarknet estrae features multi-scala, il neck PANet fonde le informazioni, e le head producono detection e keypoints.}
    \label{fig:yolo}
\end{figure}

\subsubsection{Backbone: CSPDarknet}
Estrae features gerarchiche dall'immagine di input attraverso una serie di blocchi convoluzionali con connessioni cross-stage partial (CSP). Produce feature maps a multiple risoluzioni.

\subsubsection{Neck: PANet (Path Aggregation Network)}
Fonde le feature maps provenienti da diversi livelli del backbone, permettendo al modello di rilevare oggetti di dimensioni variabili. Utilizza connessioni top-down e bottom-up.

\subsubsection{Head: Detection + Pose}
Due head parallele producono:
\begin{itemize}
    \item \textbf{Detection Head}: Bounding boxes con classe e confidence
    \item \textbf{Pose Head}: Keypoints con coordinate $(x, y)$ e visibility score
\end{itemize}

\subsection{Keypoints Anatomici (24 Punti)}

Il modello estrae 24 keypoints specifici per l'anatomia canina, suddivisi in regioni corporee:

\begin{table}[H]
\centering
\small
\begin{tabular}{clcl}
\toprule
\textbf{ID} & \textbf{Nome} & \textbf{ID} & \textbf{Nome} \\
\midrule
0 & nose & 12 & right\_front\_knee \\
1 & left\_eye & 13 & left\_front\_paw \\
2 & right\_eye & 14 & right\_front\_paw \\
3 & left\_ear\_base & 15 & left\_back\_elbow \\
4 & right\_ear\_base & 16 & right\_back\_elbow \\
5 & left\_ear\_tip & 17 & left\_back\_knee \\
6 & right\_ear\_tip & 18 & right\_back\_knee \\
7 & throat & 19 & left\_back\_paw \\
8 & withers (garrese) & 20 & right\_back\_paw \\
9 & left\_front\_elbow & 21 & tail\_start \\
10 & right\_front\_elbow & 22 & tail\_end \\
11 & left\_front\_knee & 23 & chin \\
\bottomrule
\end{tabular}
\caption{Elenco completo dei 24 keypoints anatomici estratti dal modello Dog-Pose.}
\label{tab:keypoints}
\end{table}

Ogni keypoint \`e rappresentato da una tripla $(x, y, v)$ dove:
\begin{itemize}
    \item $x, y$: coordinate in pixel nel frame originale
    \item $v \in [0, 1]$: visibility score (1 = visibile, 0 = occluso/non rilevato)
\end{itemize}

\subsection{Dataset Dog-Pose}

Il modello viene trainato sul dataset Dog-Pose fornito da Ultralytics, specificamente progettato per pose estimation canina.

\begin{table}[H]
\centering
\begin{tabular}{ll}
\toprule
\textbf{Propriet\`a} & \textbf{Valore} \\
\midrule
Immagini Training & 6,773 \\
Immagini Test & 1,703 \\
Classi & 1 (Dog) \\
Keypoints per istanza & 24 \\
Formato annotazioni & YOLO Pose (txt) \\
Fonte & Ultralytics Hub \\
\bottomrule
\end{tabular}
\caption{Statistiche del dataset Dog-Pose utilizzato per il training del backbone.}
\label{tab:dogpose_dataset}
\end{table}

\subsection{Training}

Il training utilizza transfer learning partendo dal modello \texttt{yolo11n-pose.pt} pre-trainato su human pose estimation (17 keypoints umani). Il modello viene adattato alla struttura anatomica canina (24 keypoints).

\subsubsection{Configurazione v2 (Attuale)}

\begin{table}[H]
\centering
\begin{tabular}{ll}
\toprule
\textbf{Parametro} & \textbf{Valore} \\
\midrule
Modello base & yolo11n-pose.pt \\
Epochs & 150 \\
Image size & 640 $\times$ 640 \\
Batch size & 16 \\
Optimizer & AdamW \\
Learning rate iniziale & 0.001 \\
Learning rate finale & 0.01 \\
Weight decay & 0.0005 \\
Early stopping patience & 20 epochs \\
Device & CUDA (single GPU) \\
\bottomrule
\end{tabular}
\caption{Hyperparameters per il training del backbone Dog-Pose v2.}
\label{tab:backbone_training}
\end{table}

\subsubsection{Data Augmentation}

Durante il training vengono applicate le seguenti trasformazioni:

\begin{itemize}
    \item \textbf{Mosaic}: Combina 4 immagini in una sola (prob. 1.0)
    \item \textbf{MixUp}: Sovrapposizione di due immagini (prob. 0.1)
    \item \textbf{Rotazione}: $\pm 10$\textdegree
    \item \textbf{Traslazione}: $\pm 10\%$
    \item \textbf{Scaling}: $0.5\times$ - $1.5\times$
    \item \textbf{Horizontal Flip}: 50\%
    \item \textbf{HSV Augmentation}: Variazioni di hue, saturation, value
\end{itemize}

\subsection{Output del Modello}

Per ogni frame processato, il backbone ritorna una lista di detection, ciascuna contenente:

\begin{lstlisting}[style=python, caption={Struttura output del backbone per ogni cane rilevato}]
detection = {
    'bbox': [x1, y1, x2, y2],     # Bounding box in pixel
    'confidence': 0.92,           # Confidence detection
    'keypoints': np.array(24, 3), # 24 keypoints (x, y, visibility)
    'roi': np.ndarray,            # Immagine croppata
    'class_id': 0                 # Sempre 0 (dog)
}
\end{lstlisting}

\subsection{Normalizzazione Keypoints}

Prima di essere passati al Pose Classifier, i keypoints vengono normalizzati rispetto alla bounding box per renderli invarianti a scala e posizione:

\begin{lstlisting}[style=python, caption={Normalizzazione keypoints rispetto alla bounding box}]
def normalize_keypoints(keypoints, bbox):
    x1, y1, x2, y2 = bbox
    w, h = x2 - x1, y2 - y1

    normalized = keypoints.copy()
    normalized[:, 0] = (keypoints[:, 0] - x1) / w  # x in [0, 1]
    normalized[:, 1] = (keypoints[:, 1] - y1) / h  # y in [0, 1]
    # visibility rimane invariato

    return normalized  # Shape: (24, 3)
\end{lstlisting}

\subsection{Risultati v2}

Il training v2 con 150 epochs ha prodotto risultati eccellenti, superando significativamente i target prefissati.

\begin{table}[H]
\centering
\begin{tabular}{lcc}
\toprule
\textbf{Metrica} & \textbf{Target} & \textbf{Ottenuto (v2)} \\
\midrule
mAP@0.5 & $> 0.85$ & \textbf{0.987} \\
Precision & $> 0.90$ & \textbf{0.969} \\
Recall & $> 0.90$ & \textbf{0.977} \\
Inference time (GPU) & $< 20$ms & $\sim 8$ms \\
\bottomrule
\end{tabular}
\caption{Metriche del backbone Dog-Pose v2 (150 epochs). Il modello raggiunge una mAP@0.5 del 98.7\%.}
\label{tab:backbone_metrics}
\end{table}

\begin{figure}[H]
    \centering
    \includegraphics[width=\textwidth]{figures/backbone_training_curves.png}
    \caption{Curve di training del backbone Dog-Pose v2. In alto: loss di training (box, pose, keypoint object, classification, DFL). In basso: loss di validation e metriche (precision, recall, mAP50, mAP50-95). Si nota la convergenza stabile senza overfitting.}
    \label{fig:backbone_training}
\end{figure}

\begin{figure}[H]
    \centering
    \includegraphics[width=0.9\textwidth]{figures/backbone_detection_examples.jpg}
    \caption{Esempi di detection del backbone Dog-Pose su immagini di validazione. Ogni cane \`e identificato con bounding box (blu), confidence score e 24 keypoints anatomici sovrapposti.}
    \label{fig:backbone_examples}
\end{figure}

\paragraph{Analisi:} L'elevata precision (96.9\%) indica che quasi tutte le detection sono corrette, mentre l'alto recall (97.7\%) significa che il modello rileva quasi tutti i cani presenti nel frame. Questo \`e fondamentale per il sistema ResQPet, dove ogni cane deve essere analizzato.
