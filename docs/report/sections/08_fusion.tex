% ============================================
% SEZIONE 8: SISTEMA DI FUSIONE
% ============================================

\section{Sistema di Fusione}
\label{sec:fusion}

\subsection{Obiettivo}

Il sistema di fusione combina le probabilit\`a dei quattro classificatori in un unico \textbf{Stray Index}, una metrica normalizzata in $[0, 1]$ che quantifica la probabilit\`a che un cane sia randagio.

\begin{figure}[H]
    \centering
    \resizebox{0.9\textwidth}{!}{%
        % Fusion Module - Weighted Sum
\begin{tikzpicture}[
    node distance=1cm,
    box/.style={rectangle, draw, rounded corners, minimum width=2.5cm, minimum height=0.7cm, align=center, font=\small},
    arrow/.style={->, thick, >=stealth}
]

% Inputs with weights
\node[box, fill=blue!30] (collar) {Collar: $P_{nc}$};
\node[below=0.1cm of collar, font=\scriptsize] {peso: 35\%};

\node[box, fill=green!30, below=1cm of collar] (skin) {Skin: $P_{sick}$};
\node[below=0.1cm of skin, font=\scriptsize] {peso: 25\%};

\node[box, fill=orange!30, below=1cm of skin] (pose) {Pose: $P_{stray}$};
\node[below=0.1cm of pose, font=\scriptsize] {peso: 15\%};

\node[box, fill=purple!30, below=1cm of pose] (breed) {Breed: $P_{stray}$};
\node[below=0.1cm of breed, font=\scriptsize] {peso: 10\%};

\node[box, fill=gray!30, below=1cm of breed] (context) {Context: $P_{ctx}$};
\node[below=0.1cm of context, font=\scriptsize] {peso: 15\%};

% Fusion
\node[box, fill=red!20, right=3cm of skin, minimum width=3cm, minimum height=1.2cm] (fusion) {Weighted Sum\\$\sum w_i \cdot P_i$};

% Output
\node[box, fill=red!40, right=2cm of fusion, minimum width=2.5cm] (output) {Stray Index\\$SI \in [0,1]$};

% Arrows
\draw[arrow] (collar.east) -- ++(0.5,0) |- (fusion.west);
\draw[arrow] (skin.east) -- (fusion.west);
\draw[arrow] (pose.east) -- ++(0.5,0) |- (fusion.west);
\draw[arrow] (breed.east) -- ++(1,0) |- (fusion.west);
\draw[arrow] (context.east) -- ++(1.5,0) |- (fusion.west);
\draw[arrow] (fusion) -- (output);

\end{tikzpicture}

    }%
    \caption{Schema del sistema di fusione pesata. Le quattro probabilit\`a vengono combinate attraverso una media pesata per produrre lo Stray Index finale.}
    \label{fig:fusion}
\end{figure}

\subsection{Componenti della Fusione}

Le quattro probabilit\`a in input rappresentano aspetti complementari:

\begin{table}[H]
\centering
\begin{tabular}{llcc}
\toprule
\textbf{Componente} & \textbf{Significato} & \textbf{Peso} & \textbf{Range} \\
\midrule
$P_c$ (collar) & Probabilit\`a assenza collare & 35\% & $[0, 1]$ \\
$P_s$ (skin) & Probabilit\`a malattia cutanea & 20\% & $[0, 1]$ \\
$P_p$ (pose) & Probabilit\`a postura stray-like & 25\% & $[0, 1]$ \\
$P_b$ (breed) & Prior abbandono data la razza & 20\% & $[0, 1]$ \\
\bottomrule
\end{tabular}
\caption{Componenti della fusione con relativi pesi e range.}
\label{tab:fusion_components}
\end{table}

\subsection{Formula di Fusione}

Lo Stray Index \`e calcolato come media pesata:

\begin{equation}
    \text{SI} = w_c \cdot P_c + w_s \cdot P_s + w_p \cdot P_p + w_b \cdot P_b
\end{equation}

dove $w_c = 0.35$, $w_s = 0.20$, $w_p = 0.25$, $w_b = 0.20$ e $\sum w_i = 1$.

\subsubsection{Implementazione}

\begin{lstlisting}[style=python, caption={Calcolo dello Stray Index}]
FUSION_WEIGHTS = {
    'collar': 0.35,
    'skin': 0.20,
    'pose': 0.25,
    'breed': 0.20
}

def compute_stray_index(p_collar, p_skin, p_pose, p_breed):
    stray_index = (
        FUSION_WEIGHTS['collar'] * p_collar +
        FUSION_WEIGHTS['skin'] * p_skin +
        FUSION_WEIGHTS['pose'] * p_pose +
        FUSION_WEIGHTS['breed'] * p_breed
    )
    return np.clip(stray_index, 0, 1)
\end{lstlisting}

\subsection{Giustificazione dei Pesi}

I pesi sono stati scelti in base alla rilevanza diagnostica di ciascun indicatore:

\begin{itemize}
    \item \textbf{Collar (35\%)}: L'assenza di collare \`e l'indicatore pi\`u forte e direttamente osservabile di non appartenenza. Un cane con collare \`e quasi certamente padronale.

    \item \textbf{Pose (25\%)}: La postura riflette lo stato emotivo e comportamentale del cane. Cani randagi tendono a mostrare comportamenti difensivi o sottomessi osservabili in tempo reale.

    \item \textbf{Skin (20\%)}: Le condizioni cutanee indicano il livello di cura ricevuto. Malattie non trattate suggeriscono mancanza di accesso a cure veterinarie.

    \item \textbf{Breed (20\%)}: I prior statistici forniscono informazione contestuale basata su dati reali dei canili, ma sono meno specifici per il singolo individuo.
\end{itemize}

\subsection{Classificazione Finale}

Lo Stray Index viene mappato in tre categorie semantiche:

\begin{figure}[H]
    \centering
    \resizebox{0.9\textwidth}{!}{%
        % Stray Index Visualization Bar
\begin{tikzpicture}

% Background bar
\shade[left color=green!60, right color=red!60] (0,0) rectangle (10,0.8);

% Border
\draw[thick] (0,0) rectangle (10,0.8);

% Threshold lines
\draw[dashed, thick] (3,0) -- (3,1.2);
\draw[dashed, thick] (7,0) -- (7,1.2);

% Labels
\node[below] at (0,0) {0.0};
\node[below] at (3,0) {0.3};
\node[below] at (5,0) {0.5};
\node[below] at (7,0) {0.7};
\node[below] at (10,0) {1.0};

% Zone labels
\node[above, font=\small\bfseries] at (1.5,0.8) {Padronale};
\node[above, font=\small\bfseries] at (5,0.8) {Incerto};
\node[above, font=\small\bfseries] at (8.5,0.8) {Randagio};

% Example marker
\fill[blue] (6.5,0.4) circle (0.15);
\node[above=0.5cm of {(6.5,0.4)}, font=\scriptsize] {SI = 0.65};

% Arrow
\draw[->, very thick] (-0.5,0.4) -- (10.5,0.4);
\node[right] at (10.5,0.4) {SI};

\end{tikzpicture}

    }%
    \caption{Visualizzazione delle soglie di classificazione dello Stray Index. Verde: Padronale, Giallo: Possibile smarrito, Rosso: Probabile randagio.}
    \label{fig:stray_bar}
\end{figure}

\begin{table}[H]
\centering
\begin{tabular}{ccll}
\toprule
\textbf{Range SI} & \textbf{Codice} & \textbf{Classificazione} & \textbf{Azione} \\
\midrule
$[0.0, 0.3)$ & \colorbox{green!30}{Verde} & Padronale & Nessuna \\
$[0.3, 0.7)$ & \colorbox{yellow!50}{Giallo} & Possibile Smarrito & Monitoraggio \\
$[0.7, 1.0]$ & \colorbox{red!30}{Rosso} & Probabile Randagio & Alert attivo \\
\bottomrule
\end{tabular}
\caption{Soglie di classificazione e azioni associate.}
\label{tab:stray_thresholds}
\end{table}

\subsubsection{Implementazione Classificazione}

\begin{lstlisting}[style=python, caption={Classificazione basata su Stray Index}]
THRESHOLDS = {'owned': 0.3, 'lost': 0.7}

def classify_stray_index(stray_index):
    if stray_index < THRESHOLDS['owned']:
        return {
            'status': 'PADRONALE',
            'color': '#22c55e',  # Verde
            'action': 'none',
            'confidence': 1 - stray_index / 0.3
        }
    elif stray_index < THRESHOLDS['lost']:
        return {
            'status': 'POSSIBILE_SMARRITO',
            'color': '#eab308',  # Giallo
            'action': 'monitor',
            'confidence': 1 - 2 * abs(stray_index - 0.5)
        }
    else:
        return {
            'status': 'PROBABILE_RANDAGIO',
            'color': '#ef4444',  # Rosso
            'action': 'alert',
            'confidence': (stray_index - 0.7) / 0.3
        }
\end{lstlisting}

\subsection{Analisi di Sensibilit\`a}

L'impatto di ciascun componente sullo Stray Index finale \`e stato analizzato variando i pesi:

\begin{itemize}
    \item \textbf{Collar}: Ha l'impatto maggiore. Variando $w_c$ da 0.1 a 0.5, lo SI varia significativamente
    \item \textbf{Pose}: Secondo per importanza, cattura comportamenti osservabili
    \item \textbf{Skin/Breed}: Contributi pi\`u moderati, utili per disambiguazione
\end{itemize}

\subsection{Esempio di Calcolo}

\textbf{Scenario}: Cane rilevato con i seguenti valori:

\begin{itemize}
    \item $P_c = 0.80$ (probabilmente senza collare)
    \item $P_s = 0.30$ (possibile lieve condizione cutanea)
    \item $P_p = 0.65$ (postura moderatamente stray-like)
    \item $P_b = 0.55$ (razza a media presenza nei canili)
\end{itemize}

\textbf{Calcolo}:
\begin{align}
    \text{SI} &= 0.35 \times 0.80 + 0.20 \times 0.30 + 0.25 \times 0.65 + 0.20 \times 0.55 \\
    &= 0.28 + 0.06 + 0.1625 + 0.11 \\
    &= 0.6125
\end{align}

\textbf{Risultato}: $\text{SI} = 0.61 \rightarrow$ \colorbox{yellow!50}{POSSIBILE SMARRITO}

\subsection{Gestione Valori Mancanti}

Nel caso un classificatore non sia disponibile o fallisca:

\begin{lstlisting}[style=python, caption={Gestione valori mancanti nella fusione}]
def compute_stray_index_robust(components):
    # Valori di default (massima incertezza)
    defaults = {
        'collar': 0.5, 'skin': 0.3,
        'pose': 0.5, 'breed': 0.5
    }

    # Usa valore reale o default
    values = {k: components.get(k, defaults[k]) for k in FUSION_WEIGHTS}

    # Calcola SI
    stray_index = sum(FUSION_WEIGHTS[k] * values[k] for k in FUSION_WEIGHTS)

    return stray_index
\end{lstlisting}
