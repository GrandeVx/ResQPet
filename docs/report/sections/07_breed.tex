% ============================================
% SEZIONE 7: BREED CLASSIFIER
% ============================================

\section{Breed Classifier}
\label{sec:breed}

\subsection{Ruolo nel Sistema}

Il Breed Classifier identifica la razza del cane per applicare \textbf{prior statistici} basati su dati reali dei canili italiani. Alcune razze sono statisticamente pi\`u rappresentate tra i cani abbandonati.

\begin{itemize}
    \item \textbf{Input}: ROI del cane ridimensionata a 224$\times$224 pixel
    \item \textbf{Output}: Categoria razza + $P(\text{stray}|\text{breed})$
    \item \textbf{Peso Fusion}: 20\%
\end{itemize}

\subsection{Architettura}

Il classificatore utilizza EfficientNet-B0, un'architettura ottimizzata per il trade-off accuracy/efficienza.

\begin{figure}[H]
    \centering
    \resizebox{0.7\textwidth}{!}{%
        % EfficientNet MBConv Block
\begin{tikzpicture}[
    node distance=0.7cm,
    box/.style={rectangle, draw, minimum width=2.2cm, minimum height=0.5cm, align=center, font=\small},
    arrow/.style={->, thick, >=stealth}
]

% Main path
\node[box, fill=blue!20] (input) {Input};
\node[box, fill=green!20, below=of input] (expand) {Conv 1$\times$1 (expand)};
\node[box, fill=green!20, below=of expand] (bn1) {BatchNorm + Swish};
\node[box, fill=yellow!20, below=of bn1] (dw) {DepthwiseConv 3$\times$3};
\node[box, fill=green!20, below=of dw] (bn2) {BatchNorm + Swish};
\node[box, fill=orange!20, below=of bn2] (se) {SE Block};
\node[box, fill=green!20, below=of se] (project) {Conv 1$\times$1 (project)};
\node[box, fill=green!20, below=of project] (bn3) {BatchNorm};

% Addition
\node[circle, draw, below=of bn3, minimum size=0.5cm] (add) {$+$};
\node[box, fill=blue!20, below=of add] (output) {Output};

% Arrows
\draw[arrow] (input) -- (expand);
\draw[arrow] (expand) -- (bn1);
\draw[arrow] (bn1) -- (dw);
\draw[arrow] (dw) -- (bn2);
\draw[arrow] (bn2) -- (se);
\draw[arrow] (se) -- (project);
\draw[arrow] (project) -- (bn3);
\draw[arrow] (bn3) -- (add);
\draw[arrow] (add) -- (output);

% Skip connection
\draw[arrow, dashed, red] (input.east) -- ++(1.2,0) |- (add.east);
\node[right=1.4cm of dw, font=\scriptsize, text=red] {Skip};

\end{tikzpicture}

    }%
    \caption{Mobile Inverted Bottleneck (MBConv), il blocco fondamentale di EfficientNet. Utilizza depthwise separable convolutions e squeeze-and-excitation per efficienza.}
    \label{fig:mbconv}
\end{figure}

\subsubsection{Struttura Completa}

\begin{enumerate}
    \item \textbf{Backbone}: EfficientNet-B0 (ImageNet pre-trained)
    \begin{itemize}
        \item Input: 224$\times$224$\times$3
        \item Output: 1,280 features
        \item Blocchi MBConv con depth-wise separable convolutions
    \end{itemize}
    \item \textbf{Classification Head}:
    \begin{itemize}
        \item Dropout(0.3)
        \item Linear(1280 $\rightarrow$ 256)
        \item ReLU
        \item BatchNorm1d(256)
        \item Dropout(0.3)
        \item Linear(256 $\rightarrow$ 12)
    \end{itemize}
\end{enumerate}

\begin{lstlisting}[style=python, caption={Definizione architettura Breed Classifier}]
class BreedClassifier(nn.Module):
    def __init__(self, num_classes=12):
        super().__init__()
        # EfficientNet-B0 backbone
        self.backbone = timm.create_model(
            'efficientnet_b0',
            pretrained=True,
            num_classes=0
        )
        # Custom classifier head
        self.classifier = nn.Sequential(
            nn.Dropout(0.3),
            nn.Linear(1280, 256),
            nn.ReLU(),
            nn.BatchNorm1d(256),
            nn.Dropout(0.3),
            nn.Linear(256, num_classes)
        )

    def forward(self, x):
        features = self.backbone(x)
        return self.classifier(features)

    def predict_proba(self, x):
        return torch.softmax(self.forward(x), dim=-1)
\end{lstlisting}

\subsection{Macro-Categorie Razze}

Le 120 razze originali di Stanford Dogs vengono raggruppate in 12 macro-categorie per semplificare la classificazione e permettere l'applicazione di prior significativi.

\begin{table}[H]
\centering
\small
\begin{tabular}{lp{8cm}}
\toprule
\textbf{Categoria} & \textbf{Razze Incluse} \\
\midrule
pitbull\_amstaff & American Staffordshire Terrier, Staffordshire Bull Terrier, Pit Bull \\
shepherd & German Shepherd, Belgian Malinois, Australian Shepherd, Border Collie \\
retriever & Labrador, Golden Retriever, Flat-coated Retriever \\
hound & Beagle, Basset Hound, Bloodhound, Greyhound, Dachshund \\
terrier & Yorkshire, West Highland, Scottish Terrier, Fox Terrier \\
toy & Chihuahua, Maltese, Pomeranian, Toy Poodle, Papillon \\
working & Rottweiler, Doberman, Boxer, Great Dane, Mastiff \\
spitz & Husky, Malamute, Samoyed, Akita, Chow Chow \\
bulldog & English Bulldog, French Bulldog, Boston Terrier \\
poodle & Standard Poodle, Miniature Poodle \\
mixed & Meticci, Razze non identificabili \\
unknown & Razza non determinabile con certezza \\
\bottomrule
\end{tabular}
\caption{Raggruppamento delle razze in 12 macro-categorie.}
\label{tab:breed_groups}
\end{table}

\subsection{Breed Priors}

I prior sono basati su statistiche reali dei canili italiani (fonti: ENPA, LAV):

\begin{table}[H]
\centering
\begin{tabular}{lcc}
\toprule
\textbf{Categoria} & \textbf{$P(\text{stray}|\text{breed})$} & \textbf{Motivazione} \\
\midrule
pitbull\_amstaff & 0.75 & Alta presenza nei canili \\
mixed & 0.70 & Categoria pi\`u comune tra randagi \\
hound & 0.55 & Cani da caccia spesso abbandonati \\
shepherd & 0.50 & Media presenza \\
working & 0.50 & Media presenza \\
terrier & 0.40 & Media-bassa \\
spitz & 0.35 & Bassa (razze ``esotiche'') \\
bulldog & 0.30 & Bassa (costosi) \\
poodle & 0.25 & Bassa (cani da compagnia) \\
retriever & 0.25 & Bassa (cani da famiglia) \\
toy & 0.20 & Molto bassa (costosi, dimensioni ridotte) \\
unknown & 0.50 & Default per razze non identificate \\
\bottomrule
\end{tabular}
\caption{Prior di probabilit\`a di abbandono per macro-categoria di razza.}
\label{tab:breed_priors}
\end{table}

\subsubsection{Calcolo P(stray|breed)}

\begin{lstlisting}[style=python, caption={Calcolo della probabilit\`a di abbandono data la razza}]
BREED_PRIORS = {
    'pitbull_amstaff': 0.75, 'mixed': 0.70, 'hound': 0.55,
    'shepherd': 0.50, 'working': 0.50, 'terrier': 0.40,
    'spitz': 0.35, 'bulldog': 0.30, 'poodle': 0.25,
    'retriever': 0.25, 'toy': 0.20
}

def compute_p_stray_breed(predicted_breed, confidence):
    base_prior = BREED_PRIORS.get(predicted_breed, 0.50)

    # Regolarizza verso 0.5 se confidence bassa
    p_stray = base_prior * confidence + 0.5 * (1 - confidence)

    return p_stray
\end{lstlisting}

\subsection{Dataset}

\begin{table}[H]
\centering
\begin{tabular}{ll}
\toprule
\textbf{Propriet\`a} & \textbf{Valore} \\
\midrule
Nome & Stanford Dogs Dataset \\
Fonte & Stanford Vision Lab \\
Totale immagini & $\sim$20,580 \\
Razze originali & 120 \\
Macro-categorie & 12 \\
Training & 70\% \\
Validation & 15\% \\
Test & 15\% \\
\bottomrule
\end{tabular}
\caption{Statistiche del dataset Stanford Dogs.}
\label{tab:breed_dataset}
\end{table}

\subsection{Data Augmentation}

\begin{table}[H]
\centering
\begin{tabular}{ll}
\toprule
\textbf{Trasformazione} & \textbf{Parametri} \\
\midrule
Resize & 256 $\times$ 256 \\
RandomCrop & 224 $\times$ 224 \\
Horizontal Flip & 50\% \\
Vertical Flip & 20\% \\
Rotazione & $\pm 15$\textdegree \\
ColorJitter & brightness=0.2, contrast=0.2, saturation=0.2, hue=0.1 \\
RandomAffine & translate=(0.1, 0.1), scale=(0.9, 1.1) \\
Normalize & ImageNet mean/std \\
\bottomrule
\end{tabular}
\caption{Data augmentation per il training del Breed Classifier.}
\label{tab:breed_aug}
\end{table}

\subsection{Training}

Il training utilizza learning rate differenziato per backbone e classifier:

\begin{table}[H]
\centering
\begin{tabular}{ll}
\toprule
\textbf{Parametro} & \textbf{Valore} \\
\midrule
Epochs & 30 \\
Batch size & 32 \\
Optimizer & AdamW \\
LR backbone & $10^{-5}$ (fine-tuning conservativo) \\
LR classifier & $10^{-4}$ \\
Weight decay & $10^{-4}$ \\
Loss & CrossEntropyLoss (class weights) \\
Scheduler & CosineAnnealingLR (T\_max=30) \\
Early stopping & 10 epochs \\
\bottomrule
\end{tabular}
\caption{Configurazione training del Breed Classifier.}
\label{tab:breed_training}
\end{table}

\subsection{Metriche}

\begin{table}[H]
\centering
\begin{tabular}{lcc}
\toprule
\textbf{Metrica} & \textbf{Target} & \textbf{Ottenuto} \\
\midrule
Top-1 Accuracy & $> 0.60$ & \textbf{0.863} \\
\bottomrule
\end{tabular}
\caption{Metriche di valutazione del Breed Classifier. Accuracy supera ampiamente il target.}
\label{tab:breed_metrics}
\end{table}
